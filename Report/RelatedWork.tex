\section{Related Work}
\label{sec:related work}
There has been previous research in the area of peak demand. Johnson et al. propose a system where a battery should be placed between the electricity provider's distribution grid and the customer's home \cite{johnson2011energy}. The system operates in three different modes, depending on thresholds. The system could request the household's exact electricity demand at that time. It could also request more than the demand and use the excessive electricity to charge the battery. Finally, it could request less than the current demand and fill the gap with electricity from the battery. The idea with this concept is to charge the battery whenever the household is idle and then use power from the battery when there is more activity, e.g. when cooking dinner \cite{johnson2011energy}.

The proposed method is indeed an interesting one, but it also has its difficulties. One problem is to determine how large the battery should be. The battery accounts for a significant part of the system's initial cost, a large battery is more expensive than a smaller one. On the other hand, a battery that is relatively big could result in a completely flat demand curve. Furthermore, in-depth knowledge of the prospected household's demand curve is needed to calculate the size of the battery since this varies from household to household. The goal is to find the smallest battery that can achieve the optimal peak demand. Another battery related problem is that batteries suffer from energy losses when transforming alternating current (used in the power grid and outlets) to direct current (used to store electricity in the battery). This means that the bigger the battery is, the bigger the loss is. Furthermore, today's batteries have a limited charge and discharge cycle, i.e. they lose charging capability when they are used and therefore need to be replaced after some time. This is also an incentive to choose the battery size with care. Johnson et al. also points out another issue to resolve. Since it is very hard to anticipate the future, i.e. the power demand for the next hour, it is very problematic to find efficient charge and discharge algorithms. A system that makes bad decisions about when to charge and discharge the battery is even likely to perform worse then if the system was removed. The complicated part is finding an algorithm that is not too greedy, i.e. an algorithm that does not react too fast when small changes occur in the demand curve. A greedy algorithm would simply alternate between charge and discharge for every little peak and trough in the demand. On the other hand, if the algorithm is more generous it might be too focused on lowering the maximum peak that it does not consider all other peaks that occur during a day.

Georgiadis and Papatriantafilou \cite{georgiadis2013greedy} propose a model for the problem of unforecasted energy dispatch and storage, where electric vehicles are mentioned as possible energy storage. Furthermore, a greedy online algorithm that dispatches the load and utilizes storage capabilities efficiently is suggested. 

Gaing \cite{gaing} also proposes a solution to the dispatch problem, but with focus on the economic aspects. Gaing uses a particle swarm optimization heuristic method for solving the economic dispatch in power system.
