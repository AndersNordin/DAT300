\section{Conclusion}
\label{sec:conclusion}
In this report we demonstrated that it is possible to lower the peak demand for a household without making the occupants change their habits. The components required are relatively cheap consisting of an off-the-shelf product coupled to our deployed software. Since the software is quite lightweight it is possible to implement and having it running from a small computer, e.g. a raspberry pi. The prototype in its current shape is probably more a proof of concept than an actual product. For normal consumers to apply a similar method in their homes, the components and software could be integrated into the home appliances and allowing the system to utilize the sensors and control switches that already exist in home appliances today. The scheduler could, for example, be placed inside the smart meter and no extra devices needs to be purchased. This approach could thereby be phased into people homes when they upgrade their equipment. The system could also be installed in new-built homes directly and thereby speed up the transition.

For this change to happen, the electricity pricing model needs to be changed. The electricity providers should not charge you only for how much power you consume, but also for peak usage. This would give consumers and also home appliance manufacturers an incentive to push for a change. This could help the electricity providers avoid peaks in the grid and thereby lower the greenhouse emissions while at the same time lead to lower cost of maintenance of the distribution grid.

\subsection{Future Work}
All algorithms that try to flatten peak demand make decisions without knowing what will happen next in the household. They do not know which devices that are currently active and for how long they will be so. If an algorithm could identify what devices are being used in each given time slot, it could make a more qualified guess on the change of future demand. For example, if the microwave is turned on, the algorithm knows the typical behaviour of a microwave and could therefore anticipate the future demand. Furthermore, if the stove is turned on, the algorithm might anticipate that food is about to be cooked in the household and that more kitchen appliances will be used in the near future and therefore take some action and prepare for that scenario. However, this idea has also its cost. For the scheduler to know which appliances that are being used it would need to have communication with all interactive loads as well as all the background loads. This would require all manufacturers of home appliances to include such communication. One other solution is to let the scheduler perform thorough analysis of the active loads and compare it with known consumption traces of home appliances to guess what appliances are active. A problem with this solution is that one appliance does not behave exactly as one of a similar model, or as it did yesterday. Microwave ovens, for example, comes in different sizes with different features. Some have a max power of 800 watts while some have 1000 watts. There is also possible for the user to select the power level each time the microwave oven is used. This means that each microwave oven may have numerous consumption traces which, furthermore, differentiates from a microwave oven from another brand and model. In order to keep track of all these traces, a quite large database would be required. Another challenge with this kind of analysis is that it is uncertain. The scheduler can only measure the total consumption of all active loads. There might be several different scenarios that yield in the same total consumption and the scheduler will have no way to differentiate these. The total consumption of the toaster and the microwave may for example be equal to the consumption of the stove.

Another interesting idea is to check the efficiency of the algorithm (LSF) when applied on a larger scale, not just one household but a whole neighbourhood or a smaller city. When applying the algorithm on a larger scale one must also deal with new issues like privacy, how households can communicate without violating the privacy of the occupants.